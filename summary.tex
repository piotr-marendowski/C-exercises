\documentclass[openany]{book}

\usepackage[T1]{fontenc}
\usepackage[utf8]{inputenc}
\usepackage{listings, lstautogobble}            % code segments
\usepackage{array, multirow}                    % tables multirows
\usepackage[table]{xcolor}                      % tables colors
\usepackage[nomarginpar, margin=3cm]{geometry}  % for margins to be equal size
\usepackage{listings, lipsum, courier}          % fonts

% inverts colors of the document, I use it mostly for my eyes' sake in night
\pagecolor[rgb]{0.2, 0.2, 0.2}
\color[rgb]{1, 1, 1}

% set code segment's parameters
\lstset{language=C,
    basicstyle=\small\ttfamily,
    stringstyle=\ttfamily,
    showstringspaces=false,
    autogobble=true
}

\newcommand\tab[1][20px]{\hspace*{#1}}      % tabs 20px
\setlength{\parindent}{0pt}                 % no tabs at start of the paragraphs

\title{The summary of K.N. King's "C Programming A Modern Approach" 2nd version}
\author{Piotr Marendowski}
\date{March 2023}

\begin{document}
    \maketitle

    \chapter{Note}
    The summary of K.N. King's "C Programming A Modern Approach" 2nd version

    \tableofcontents



    \chapter{C Fundamentals}
    
    \section{Steps of Executing a C Program}
    Automated proccess:
    \begin{enumerate}
        \item\textbf{Preprocessing} - Preprocessor is executing directives (they begin with \#).
        \item\textbf{Compiling} - Compiler translates program into machine instructions (object code).
        \item\textbf{Linking} - Linker combines object code and code needed for execution of the program.
    \end{enumerate}

    \section{The General Form of a Simple Program} Simple C programs have this form:

    \bigskip
    \textit{directives}
    \smallskip

    \small\texttt{int main(void)} \\
    \small\texttt{\{} \\
    \tab\textit{statements} \\
    \small\texttt{\}}
    \bigskip

    \textbf{Directives} - Begin with '\#' symbol, they state what headers include to program.

    \textbf{Functions} - They are segments of code that take arguments, and returns (or not) a value. Only \texttt{main} function is required.

    \textbf{Statements} - Commands to execute, mostly end with semicolon.

    \bigskip
    \textbf{String literal} - Series of characters enclosed in double quotation marks, e.g. \texttt{"Hello world!"}.

    \textbf{New-line character} - \texttt{\textbackslash n} is an escape sequence, which advances to the next line of the output.

    \textbf{Comments} - Are ommited in program execution, can be used to comment single line e.g. \texttt{/* Comment */}, or block of lines. From C99 we can use one line comments e.g. \texttt{// Comment}.

    \section{Variables and Assigments}

    \textbf{Variable} - Place to store calculation's output, for the future use.
    Variable's characteristics:
    \begin{itemize}
        \item 
        \textbf{Types} - For now, there are two types of variables:
        \begin{itemize}
            \item \texttt{int} - Integer types, can store quite big whole numbers, but that depends on your computer's architecture.
            \item \texttt{float} - Can store bigger numbers, as well as digits after the decimal point.
        \end{itemize}

        \item 
        \textbf{Declarations} - To use a variable, we first need to declare it. It means that we need to specify variable's type, and name. We can chain declarations with the same type e.g. \texttt{int i, sum, x;}. In C99 they can now be declared after statements, unlike in C89.
        
        \item
        \textbf{Assignment} - Assigns value to a variable. Variable is on the left side, while value, expression, formula etc. is on the right side. To assign something to a variable, we first need to declare it. Examples:
        \begin{lstlisting}
        int i;
        float f;
        i = 1;
        f = 1.5;
        \end{lstlisting}
    \end{itemize}

    \subsection*{Initialization}
    At the default most variables are uninitialized, which means that they have some random - garbage value assigned to them. In declaration we can assign value to a variable, making it an \textbf{initializer}, e.g. \texttt{int i = 0;}.

    \section{Reading Input}
    For reading input we need to use the \texttt{scanf} function, which needs a format string and value to read, e.g. \texttt{scanf("\%d", \&i);}.

    \section{Defining Names for Constants}
    To define a constant, we need to use a \textbf{macro definition}, which is interpreted by the preprocessor e.g. \texttt{\#define WIDTH 20}.

    \section{Identifiers}
    Names in C are called \textbf{identifiers}. They can begin with the lower-case, upper-case letters or underscores e.g. \texttt{times10  my\_var  \_done}. They cannot begin with a number e.g. \texttt{10times}. They cannot contain minus signs e.g. \texttt{my-var}.

    \subsection*{Keywords}
    There are number of keywords, which are prohibited from using as identifiers.

    \section{Layout of the C Program}
    We can slice C statements into \textbf{tokens}:
    \begin{lstlisting}
    printf   (   "Height:   %d\n"   ,   height   )   ;
      1      2      3         4     5     6      7   8
    \end{lstlisting}
    Tokens 1 and 2 are identifiers, token 3 is a string literal and tokens 2, 4, 6, and 7 are punctuation. Most of the time, we could put many spaces between them. But we cannot put spaces within tokens e.g. \\
    \texttt{fl oat f;}. 



    \chapter{Formatted Input/Output}

    \section{The \texttt{printf} Function}
    Needs a format string and arguments to insert there. There is no limit how much of arguments could be there. Format string could have conversion specifications, which are supplied by its arguments to insert into format string e.g. \texttt{printf("Value: \%d", i);} where \texttt{\%d} is a conversion specification and \texttt{i} is a value to be supplied to the format string.

    \section{Conversion Specifications}
    Outside of letter(s) specifying which type to covert to, they consist of the \textbf{Minimal Field Width (m)} and \textbf{Precision (p)}. They have the form of: \texttt{\%m.pX}.

    \subsection*{Minimal field width}
    Specifies the minimum number of characters to print. If this number is less than specified, then the number is right justified with spaces added. If the number of characters is greater than specified, then it will automatically expand to display all of the characters.

    \subsection*{Precision}
    Depends on the type to be displayed, reference the book for more detailed preview. 
    \newpage

    \subsection*{Conversion specifications:}
    \begin{itemize}
        \item \texttt{d} - Displays an integer in a decimal (base 10) form. \textit{p} indicates the minimum number of digits to display.
        \item \texttt{e} - Displays a floating-point number in the exponential format. \textit{p} indicates the number of digits after the decimal point.
        \item \texttt{f} - Displays a floating-point number without an exponent. \textit{p} has the same meaning as previous.
        \item \texttt{g} - Displays a floating-point number in exponential format or fixed (without an exponent). \textit{p} indicates the maximum number of significant digits to be displayed. It depends on the size of the number.
    \end{itemize}

    \subsection*{Escape sequences}
    They are characters, that would introduce problems in compilation or have some action to do e.g. insert new line. Few of them are:
    \begin{itemize}
        \item \texttt{\textbackslash a} - alert (bell),
        \item \texttt{\textbackslash b} - backspace,
        \item \texttt{\textbackslash n} - new line,
        \item \texttt{\textbackslash t} - tab,
        \item \texttt{\textbackslash "} - qoute character,
        \item \texttt{\textbackslash\textbackslash} - slash character.
    \end{itemize}

    \section{The \texttt{scanf} Function}
    This funcion handles input from the standard input stream (keyboard),  have a format string and may contain conversion specifications. The scanf call may look like that: \texttt{scanf("\%d", \&i);}. Scanf when reading an input ignores the white-space characters. It only matches input to the provided variables. If a reading error occurs, scanf will return immediately, ignoring the rest of the format string. It does not read the new-line character at the end of the input. If character cannot be read, function puts it back for the next variable and adds it to that.

    \subsection*{Ordinary Characters in Format Strings}
    We can put white-space characters into the format string, then \texttt{scanf} will read any number of white-space characters and discard them. When it encounters a non-white-space character it tries to match it with an inputted character. If it fails, it returns this variable without assigning to it anything.



    \chapter{Expressions}

    \section{Arithmetic Operators}
    \begin{tabular}{ |c|c|c| } 
        \hline
        \cellcolor{lightgray}Unary & \multicolumn{2}{|c|}{\cellcolor{lightgray}Binary} \\
        \hline
        & \cellcolor{lightgray}Additive & \cellcolor{lightgray}Multiplicative \\
        \cline{2-3}
        {}{}{\shortstack{+ unary plus \\ - unary minus}} & + addition & * multiplication \\ 
         & - subtraction & / division \\ 
         & & \% reminder \\
        \hline
    \end{tabular}

    \subsection*{Operator Precendence and Associativity}
    \textbf{Operator precendece} is in what order C calculates expressions. The arithmetic operators have the following relative precendence: \\
    \tab Highest: + - (unary) \\
    \tab\tab\tab \ * \ / \ \% \\    % single backslashes are spaces
    \tab Lowest: \ + - (binary)

    \bigskip
    \textbf{Associativity} decides in what order operators with the same precendece are calculated. The binary operators are all left associative, whilst the unary operators are all right associative.

    \section{Assignment Operators}
    Are used to store a computed value of the expression.

    \subsection*{Simple Assignment}
    Evaluates an expression, which then assigns into the variable, expression can be a \textit{constant} (always has the same value). If they do not have the same type, the value of an expression is converted to the type of the variable. Assignments can be chained together e.g. \texttt{i = j = k = 0;}. The \texttt{=} operator is right associative. 

    \subsection*{Lvalues}
    Assignment operator requires on its left side a variable, not an expression e.g. \texttt{i + j = 0;} is wrong.

    \subsection*{Compound Assignment}
    We can shorten statements e.g. \texttt{i += 2;} is equivalent to \texttt{i = i + 2;}. It works with the other operators including the following: \texttt{-= *= /= \%=}, they all work in the same way and are right associative.

    \section{Increment and Decrement Operators}
    Used to even more shorten a compound addition and subtraction by 1. E.g. 
    \begin{lstlisting}
        i = i + 1;
        j = j - 1;
    \end{lstlisting}
    Are the same as:
    \begin{lstlisting}
        i += 1;
        j -= 1;
    \end{lstlisting}
    Which are the same as:
    \begin{lstlisting}
        i++;
        j--;
    \end{lstlisting}

    % it seems that \texttt need one more minus, don't know why
    They contain a side effects - after adding or subtracting 1, values of their operands are modified. There are two types of these operators: \textbf{prefix} (\texttt{++i} or \texttt{---i}) which increments the variable first, then assigns value to \texttt{i}, and \textbf{postfix} (\texttt{i++} or \texttt{i---}) which first assings value to \texttt{i} and increments \texttt{i} after this statement.



    \chapter{Selection Statements}
    We could group most statements in this three categories:
    \begin{itemize}
        \item \textbf{Selection statements} - Test provided condition, and execute code within condition's borders, e.g. \texttt{if} and \texttt{switch} statements.
        \item \textbf{Iteration statements} - Iterate over and over again, until the condition is not true, e.g. \texttt{for, while}, and \texttt{do while} statements.
        \item \textbf{Jump statements} - They control the flow of the program, can stop iterations, skip through them or jump to any place in the program, e.g. \texttt{break},
        \texttt{continue} and \texttt{goto} statements.
    \end{itemize} 

    \section{Logical Expressions}
    \subsection*{Relational Operators}
    They are used to compare expressions, yelding 0 if statement is not true and 1 if it is.

    \bigskip
    \begin{tabular}{|@{} c l|}  % will do the trick :)
        \hline
        \textbf{ Symbol} & \textbf{Meaning} \\
        \hline
        < & less than \\
        > & greater than \\
        <= & less than or equal to \\
        >= & greater than or equal to \\
        \hline
    \end{tabular} 

    \subsection*{Equality Operators}

    \bigskip
    \begin{tabular}{|@{} c l|}
        \hline
        \textbf{ Symbol} & \textbf{Meaning} \\
        \hline
        == & equal to \\
        != & not equal to \\
        \hline
    \end{tabular}

    \subsection*{Logical Operators}

    \bigskip
    \begin{tabular}{|@{} c l l|}
        \hline
        \textbf{ Symbol} & \textbf{Meaning} & \textbf{Operation} \\
        \hline
        ! & logical negation & Inverse - if false, returns 1\\
        \&\& & logical end & If both expressions are true, returns 1 \\
        || & logical or & If either one of them them is true, returns 1 \\ 
        \hline
    \end{tabular}

    \section{The \texttt{If} Statement}
    Has the form of: \texttt{if ( expression ) statement}. If evaluated expression has a non-zero value, then statement after parentheses is executed.

    \subsection*{Compound Statements}
    We can "stack" multiple statements between the parentheses.

    \subsection*{The \texttt{else} clause}
    If we want to execute statements when our expression is not true, we need to use a \texttt{else} clause. It has the following form: \texttt{if ( expression ) statement else statement}.

    \subsection*{Cascaded \texttt{if} Statements}
    Thanks to them we can test multiple conditions. They have the following form:

    \texttt{if ( expression ) statement else if} \texttt{( expression ) statement else} \\
    \texttt{statement}.

    \subsection*{Conditional Expressions}
    Has the following form: \texttt{epr1 ? epr2 : epr3}. Tests whether the first expression is true, if it is then executes the second expression, otherwise executes the third expression.

    \subsection{Boolean Values}
    \subsubsection{C89}
    There is not a boolean type in C89, but we could declare a macro definition named \texttt{TRUE or FALSE}, e.g. \texttt{\#define TRUE 1}.

    \subsubsection*{C99}
    With the arrival of C99 we could declare \texttt{\_Bool} type, e.g. \texttt{\_Bool flag = true;}. For using this type we must declare an \texttt{\#include <stdbool.h>} directive.
    
    \section{The \texttt{switch} statement}
    Switch statement can be a better-looking and faster alternative to the cascaded
    \texttt{if}. It has the following form:
    \begin{lstlisting}
        switch ( expression ) {
            case constant-expression : statements
            ...
            case constant-expression : statements
            default : statements
        }
    \end{lstlisting}
    
    Components of the \texttt{switch} statement:
    \begin{itemize}
        \item \textbf{Controlling expression} - could be an \texttt{int} or a
        \texttt{char}, but not \texttt{float} and strings.
        \item \textbf{Case lebels} - Form: \texttt{case constant-expression :} Cannot contain variables and function calls.
        \item \textbf{Statements} - Come after each case label.
    \end{itemize}
    \texttt{Switch} statement does not need a \texttt{default} case. We use a \texttt{break} statement to stop at one choice, otherwise switch would execute all remaining cases.



    \chapter{Loops}
    Loop is a repeatedly executing statements until the controlling expression is not true.
    
    \section{The \texttt{while} Statement}
    It has the following form: \texttt{while ( expression ) statement}. First it tests the controlling expression, then executes the loop body, if the expression is false, the loop terminates. It is possible that the loop body would not be executed at all, because it could be a false from the beginning.
    
    \section{The \texttt{do} Statement}
    It resembles the \texttt{while} loop, but it tests the controlling expression after each execution of the loop body, which makes it execute always at least once. Has the following form: \\
    \texttt{do statement while ( expression ) ;}
    
    \section{The \texttt{for} Statement}
    Is used in a variety of ways, is ideal for counting loops. Has the following form: \texttt{for ( expr1 ; expr2 ; expr3 ) statement}. Every expression has its own role, in the first you could initialize or assign values to variables. In the second you place a condition, which is checked every execution and while it is true the loop will execute. In the third you put an operation which is performed at the end of each loop iteration. We could make more expressions (separated by the commas), but would need to place them in place according to their characteristics. Most of the \texttt{for} loops can be replaced by the \texttt{while} loop. We could omit any expression, but we need to compensate for it before or later. From C99 we could declare variables in the loop body, which will not be used as the variables already declared.
    

    \section{Exiting from a Loop}

    \subsection{The \texttt{break} Statement}
    Is used to jump out of the loop body, terminating it. Can escape only one level of nesting.

    \subsection{The \texttt{continue} Statement}
    Only used in loops. Transfers control to the end of the loop, but it stays in the loop for the next execution. Can be considered a "skip to the end" statement.

    \subsection*{The \texttt{goto} Statement}
    Jumps (transfers control) from its declaration to the label in the other part of the program. Label has the following form: \texttt{identifier : statement}. The \texttt{goto} statement has this form: \texttt{goto identifier ;}. Nowadays it is not commonly used, in favour of the \texttt{break, continue} and \texttt{return} statements.

    \section{The \texttt{null} Statement}
    We could omit the loop body (moving the loop body into expressions) or one of its expressions (which will create an infinite loop), e.g. \texttt{for (i = 0; ; i++) {...} }.



    \chapter{Basic Types}

    \section{Integer Types}
    They are the whole numbers. We can divide them into two categories: \texttt{signed} and \texttt{unsigned}. The \texttt{signed} integers can be both positive and negative, but because of this they maximally can only hold two times less than the \texttt{unsigned} type, which cannot contain negative numbers, thus making it one bit bigger. Sometimes we need bigger numbers, then we would use a \texttt{long} integer which on the most machines is double the \texttt{int} type. We could also store smaller integer values in the \texttt{short int} type. Every type can be \texttt{signed} or \texttt{unsigned}, e.g. \texttt{unsigned short int}. We could abbreviate names by dropping the word \texttt{int}. Types are not required to have a certain bounds or maximum values, which varies from one architecture to another. For our machine's ranges we could see the \texttt{<limits.h>} header.

    \subsection*{Integer Types in C99}
    From C99 we could now declare type \texttt{long long int}, which can be even two times the \texttt{long int} type.

    \subsection*{Integer Constants}
    Constant numbers are numbers that cannot change. Could be of the base 10, 8 and 16:
    \begin{itemize}
        \item \textbf{Decimal} - Contain digits 0 through 9, cannot begin with a zero.
        \item \textbf{Octal} - Contain digits 0 through 7, must begin with a zero.
        \item \textbf{Hexadecimal} - Containg digits 0 through 9 and letter A to F, always begin with a 0x. Letters can be either lower case or upper case.
    \end{itemize}

    We could mix together any of these without any repercussions. As well as we could indicate that constant is for example \texttt{long int} - by adding \texttt{L or l} to the number. To indicate that constant is a \texttt{unsigned int} - add \texttt{U or u}. Both can be added together. In C99 we could specify the type \texttt{long long int} by adding \texttt{LL or ll} to the end.

    \bigskip
    If the result of the arithmetic operation is more than what desired type can store, we say that occured an \textbf{Integer Overflow}.

    \subsection*{Reading and Writing Integers}
    For reading/writing operations we need to use the conversion specification for Unsigned types:
    \begin{itemize}
        \item \textbf{Unsigned} - \texttt{U} in decimal.
        \item \textbf{Octal} - \texttt{O}.
        \item \textbf{Hexadecimal} - \texttt{X}.
    \end{itemize} 
    We use could use them with type conversion specifications by appending this letter to the front:
    \begin{itemize}
        \item \textbf{Short} - \texttt{H}.
        \item \textbf{Long} - \texttt{L}.
        \item \textbf{Long Long} (C99) - \texttt{LL}.
    \end{itemize}

    \section{Floating Types}
    Numbers in the decimal form. There are three of them:
    \begin{itemize}
        \item \textbf{Float} - Single precision.
        \item \textbf{Double} - Double precision.
        \item \textbf{Long double} - Extended precision.
    \end{itemize}
    Mostly are stored according to the IEEE Floating-Point Standard.

    \subsection*{Floating Constants}
    A floating constant must contain a decimal point and/or an exponent. The exponent (if present) must be preceded by the letters E or e. We could put the letters F or f (for \texttt{float}) or LF or lf (for \texttt{double}) at the end of the number to set the desired type.

    \subsection*{Reading and Writing Floating-Point Numbers}
    In order to read a \texttt{double} value we use \texttt{lf, le, lg} (for \texttt{printf} without the l character). For the \texttt{long double} we used \texttt{Lf, Le, Lg}.

    \section{Character Types}
    For one character we use the type \texttt{char} which on the most machines corresponds to the ASCII table.

    \subsection*{Operations on Characters}
    Because of the fact that the \texttt{char} type is a really short int, we could take characters from the ASCII table and print them according to their numbers e.g. number 97 will be the character 'A'. We could also do some calculations on characters, for example by adding 1 to previous character we would get 'B'. Because of this characteristic we could compare characters and check if a number is for example greater than 'A' and less then 'Z'.

    \subsection*{Signed and Unsigned Characters}
    The C standard does not state that the \texttt{char} type is explicitly a \texttt{signed} or an \texttt{unsigned} type. Most of the time we do not care about it.

    \subsection*{Escape Sequences}
    We use them for non-printable characters. To get them all we need to use a numeric escapes which we could write in octal or hexadecimal:
    \begin{itemize}
        \item \textbf{Octal} - Does not begin with a zero, but with a backslash, after it  comes a number from the ASCII character set in octal.
        \item \textbf{Hexadecimal} - Consists of the \textbackslash x followed by the hexadecimal number.
    \end{itemize}
    An escape sequence must be enclosed in single qoutes e.g. \texttt{'\textbackslash 33'}.

    \subsection*{Operations on Character types}
    To read or write a character type, we need to use the \texttt{\%c} conversion specification. To skip white spaces before reading a variable in the \texttt{scanf} function we would need to write it like that: \texttt{" \%c"}.

    \bigskip

    We could also use the \texttt{putchar} function in order to write a single character. As well as \texttt{getchar} to read one character. They are generally faster than \texttt{scanf} or \texttt{printf} function, because they are designed to only read one variable type and thus are smaller. If there is a \texttt{scanf} it will leave peaked (not assigned) variables and than calling the \texttt{getchar} will read them.

    \section{Type Conversion}
    If we mix different types in an arithmetic expression, the compiler will convert them to the same type and then calculate them. These conversions are called \textbf{implicit expressions}, because they are handled automatically. There are also \textbf{explicit expressions} which we could work with using the cast operator.

    \subsection*{Casting}
    We could state that variable must convert to the desired type. It has the following form: \texttt{( type-name ) expression}. We use this to avoid overflowing, when one variable is converted to the smaller type than it can handle.

    \section{Type Definitions}
    Example: \texttt{typedef int Bool;} would make a \texttt{Bool} type which has the same characteristics as the \texttt{int} type.

    \section{The \texttt{sizeof} Operator}
    \texttt{sizeof ( type-name )} represents the number of bytes required to store a value belonging to type-name. Has type \texttt{size\_t} which is unsigned integer type.



    \chapter{Arrays}

    \section{One-Dimensional Array}
    A data structure containing number of elements of the same type. Elements are arranged in a single row one after another, beginning with a zero element and ending with a n-1 element because of that. \texttt{int a[10];} declares one-dimensional array with 10 \texttt{int} elements. We use an array  \textbf{subscripting} or \textbf{indexing} in order to access a particular element of the array, e.g. \texttt{a[0] = 10;}.

    \subsection*{Array Initialization}
    \texttt{int a[3] = \{1, 2, 3\};} \\
    If we initialize less than total number of elements, the rest will be zeroed. Thanks to that we could initialize all elements to zero: \texttt{int a[3] = \{0\}}. We could also omitt the length of the array if the initializer is present.

    \section{Multidimensional Arrays}
    We could create arrays with any number of dimensions:
    \begin{lstlisting}
    int m[2][2] = {{1, 2}
                   {3, 4}};
    \end{lstlisting}
    Which will have two rows and colums. They are stored in the row-major order (one row after another in a continuous block of memory). We could also declare them to be constant with a keyword \texttt{const} to not permit any modification of them.

    \section{Variable-Length Arrays}
    C99 feature, which could be used to supply a non-constant number to the declaration. For example we could first ask to specify a length and then declare array with that length, by writing in the place of the array's lenght, that variable.



    \chapter{Functions}

    \section{Defining and Calling Functions}
    \begin{lstlisting}
    double average(double a, double b)
    {
        return (a + b) / 2;
    }
    \end{lstlisting}
    \texttt{double} is the \textbf{return type} of the function, \texttt{a} and \texttt{b} are \textbf{parameters} supplied from the call of the function to be used in this function. The \texttt{return} statement is in the loop \textbf{body} which will be executed. To call this function we need to enter a function name followed by the list of \textbf{arguments}: \texttt{average(x, y);}.

    \subsection{Function Definitions}
    General function definition:
    \begin{lstlisting}
    return-type function-name ( parameters )
    {
        declarations
        statements
    }
    \end{lstlisting}
    Functions cannot return arrays. If \texttt{return-type} is \texttt{void} function doesn't return a value. If there is not a return value, type is \texttt{int} in C89, while in C99 it is illegal. If function has zero parameters it should contain the keyword \texttt{void}. Void function's body can be empty.

    \subsection*{Function Calls}
    Consists of a function name and parameters enclosed in the parentheses (if there are not any parameters, we need to write parentheses without anything inside).

    \section{Function Declarations}
    A first line of the function and is used for providing information about function which could be written as a whole below the \texttt{main} function. It must be consistent with the function's definition. They are known as \textbf{function prototypes}. 

    \section{Arguments}
    Parameters are in function definitions and arguments are in function calls. Arguments are passed by value which means that they won't be affected after function execution. 

    \subsection*{Array Arguments}
    When supplying an array compiler does not know its length so we need to add a length argument. In function calls we only pass the array name as an argument. In contradiction we can modify an array when passing it into the function.

    \subsection*{Variable-Length Array Parameters}
    To use VLAs as parameters we need to first specify parameter which will go into this VLA.

    \subsection*{Compound Literals}
    \begin{lstlisting}
    total = sum_array((int []){3, 0, 3, 4, 1}, 5);
    \end{lstlisting}
    Makes this array "on the fly" to be supplied into \texttt{sum\_array} function.

    \section{The \texttt{return} Statement}
    \texttt{return expression ;} \\
    If there is for example a \texttt{double} variable in the \texttt{return} statement, it will be converted to the function's return type. For \texttt{void} functions \texttt{return} statement is not necessary.

    \section{Program Termination}
    The \texttt{main} function should return 0 if terminated successfully.

    \subsection*{The \texttt{exit} Function}
    \texttt{exit(0)} or \texttt{exit(EXIT\_SUCCESS)} - normal termination, \texttt{exit(1)} or \texttt{exit(EXIT\_FAILURE)} indicates abnormal termination. This function and macros are declared in \texttt{stdlib.h} header.

    \section{Recursion}
    A function is recusive if it calls itself.



    \chapter{Program Organization}

    \section{Local Variables}
    Variable declared in the body of a function is said to be local to the function, which means that it cannot be seen by other functions and used by them. It has an \textbf{Automatic storage duration} which means that it will be automatically allocated and deallocated at the function's return. It also has a \textbf{block scope} which means that it can only be referenced inside this function.

    \subsection*{Static Local Variables}
    Putting the word \texttt{static} causes it to have static storage duration which means that it retains its value throughout the whole program execution. But it is still hidden from other functions.

    \section{External Variables}
    Are declared outside the body of any function. Have the static storage duration and the file scope. There are many dangers of using it e.g:
    \begin{itemize}
        \item If we would change its type we would need to check every function using it.
        \item If it will have assigned an incorrect value there will be a problem to identify the guilty function.
        \item Functions using it are hard to reuse in other programs.
    \end{itemize}
    
    \section{Blocks}
    We could declare a variable in e.g. an \texttt{if} statement, then this variable will have the block scope and will not be able to be referenced outside this scope.
    
    \section{Scope}
    In a C program, the same identifier may have several different meanings. When a declaration inside a block names an identifier that's already visible, the new declaration temporarily "hides" the old one, and the identifier takes on a new meaning. At the end of the block, the identifier regains its old meaning. If you go deeper, the most relevant variable will be used. There are file and block scopes.
    
    \section{Organizing a C Program}
    Rules to organize a program:
    \begin{itemize}
        \item A preprocessing directive does not take effect until the line on which it appears.
        \item A type name cannot be used until it has been defined.
        \item A variable cannot be used until it is declared.
    \end{itemize}



    \chapter{Pointers}

    \section{Pointer Variables}
    Every byte has a unique memory address. If a variable consists of more than one byte, then its address is the first byte occupied by it. Pointer variables "point" (store) to this address in memory.

    \subsection*{Declaring Pointer Variables}
    We declare it by adding an asterisk by the name: \texttt{int *p;} which points only to the \texttt{int} variables. Pointers can point to any type or a block in memory.

    \section{The Address and Indirection Operators}
    If \texttt{x} is a variable, then its address is \texttt{\&x}. To gain access to the object (value) that a pointer points to, we use the \texttt{*} (indirection) operator.

    \subsection*{The Address Operator}
    We could initialize pointer in declaration to point to some value e.g. \texttt{int i, p = \&i;}.

    \subsection*{The Indirection Operator}
    We could access the value pointed to e.g. \texttt{printf("\%d\textbackslash n", *p);} will display the value of \texttt{i} not its address. By changing the value of \texttt{p} we will change the value \texttt{i}.

    \section{Pointer Assignment}
    \texttt{int p = \&i;} - copies the address of i into p. \texttt{q = p} copies the contents of p (the address of i) into q, so now q points to i. We could change i by chaging q and p.

    \section{Pointers as Arguments}
    Pointers can be used as function's arguments (as aliases to variables in calling function) and be used to modify values inside the function and store them outside. We could declare p to be constant \texttt{const p;} which means that it cannot be changed.

    \section{Pointers as Return Values}
    Pointers (as aliases to other variables or external variables or static variables) could also return from functions. 



    \chapter{Pointers and Arrays}

    \section{Pointer Arithmetic}
    \begin{lstlisting}
    int a[10], *p;
    p = &a[10];
    p = 5;
    \end{lstlisting}
    p points to first element in array and assigns to it 5.

    \subsection*{Adding an Integer to a Pointer}
    \begin{lstlisting}
    int a[10], p, q, i;
    p = &a[2];      /* p points to the 2nd place */
    q = p + 3;      /* q points to 5th (p + 3) */
    p += 6;         /* p points to 8th (2 + 6) */
    \end{lstlisting}
    Is equivalent to \texttt{a[i + j];}.

    \subsection*{Subtracting an Integer to a Pointer}
    \begin{lstlisting}
    int a[10], p, *q, i;
    p = &a[8];      /* p points to 8th place */
    q = p - 3;      /* q points to 5th (p - 3) */
    p -= 6;         /* p points to the 2nd (8 - 6) */
    \end{lstlisting}
    Is equivalent to \texttt{a[i - j]}.

    \subsection*{Subtracting One Pointer from Another}
    The result is distance between the pointers.
    \begin{lstlisting}
    p = &a[5];
    q = &a[1];

    i = p - q;    /* i is 4 */
    i = q - p;    /* i is -4 */
    \end{lstlisting}

    \section{Using Pointers for Array Processing}
    We could iterate through the whole array by incrementing the pointer itself.

    \section{Using an Array Name as a Pointer}
    Array subscripting can be viewed as a form of pointer arithmetic. We cannot directly change where one element points to, but we could create a new pointer to this array and make this element point elsewhere.
    \begin{lstlisting}
    int a[10];
    a = 7;        /* stores 7 in a[0] is equivalent to &a[0] */
    (a+1) = 12;   /* stores 12 in a[1] is equivalent to a[1] */

    /* easier way to calculate sum */
    for (p = a; p < a + N; p++)
        sum +=p;
    \end{lstlisting}
    Arrays passed to functions always are treated as pointers and are not copied in contrast to that array is
    not protected agains change. Ordinary values are copied, but arrays are not. We could pass a slice of the
    whole array to a function. We could also create new pointer which points to an array and perform pointer
    arithmetic on it which will be as we operated on this array itself.

    \section{Pointers and Multidimensional Arrays}
    Arrays are stored in row-major order, which means that no matter how many dimensions they have, they all are
    stored in one big line, which could be operated through pointer arithmetic. 

    \section{Pointers and Variable-Lenght Arrays (C99)}
    Pointers can point to VLAs (any dimensions). With multi-dimensional arrays pointer need to have the type of
    the last dimension \texttt{int a[m][n], (*p)[n];}.



    \chapter{Strings}

    \section{String Literals}
    A series of characters enclosed within double quotes e.g. \texttt{"Hello, World!"}.

    \subsection*{How Strings Literals Are Stored}
    C treats string literals as character arrays. C compiler sets n + 1 characters for literal and adds a \textbf{null character} (\texttt{\textbackslash 0} escape sequence) at the end which is used to indicate end of this literal. So compiler treats it like a \texttt{char} type.

    \section{String Variables}
    We need to declare a string one character longer to leave space for the null character at the
    end. Initialize a string variable: \\
    \texttt{char date[8] = "June 14";} \\
    Where \texttt{"June 14"} is not a string literal, but an initialization of string \texttt{date}. If initializer is shorter than number of elements in the array, the leftover elements will be null.
    \bigskip

    If there is no room for string, it will be cut and no null character will be assigned (meaning that it will not be usable as a string). We could also omit setting the initializer, then the compiler will calculate length of the array for us (adding the null character at the end), but this length will be fixed and cannot be changed.

    \subsection*{Character Arrays versus Character Pointers}
    \begin{lstlisting}
    char date1[] = "June 14";
    char date2 = "June 14";
    \end{lstlisting}

    \texttt{date1} is an array, while \texttt{date2} is a pointer. Array elements can be modified, but string literal pointed to by \texttt{date2} cannot. \texttt{date1} is an array name, while \texttt{date2} is a pointer which could point to other strings during program execution.

    \section{Reading and Writing Strings}
    \subsection*{Writing Strings Using \texttt{printf} and \texttt{puts}}
    \begin{lstlisting}
    printf("%s\n", str);
    puts(str);
    \end{lstlisting}

    \texttt{printf} using the \texttt{\%s} conversion specification reads characters one by one until it finds the null character (if not found will continue to read out of bounds memory locations). \texttt{puts} after reading a string always advances to the next line (by printing a new-line character).

    \subsection*{Reading Strings Using \texttt{scanf} and \texttt{gets}}
    \begin{lstlisting}
    scanf("%s", str);
    gets(str);
    \end{lstlisting}

    \texttt{str} in \texttt{scanf} is treated like a pointer so there is no need for the address operator (\texttt{\&}). When it is called it discards white-space characters and writes into \texttt{str} every character until it encounters a white-space, will always store a null character at the end. A new-line, space or a tab character in the middle of input will cause  \texttt{scanf} to stop reading. \texttt{gets} reads the whole line of input and stores a null character at the end, it does not skip white-spaces and reads until it finds a null character. There is a better and safer alternative of \texttt{gets} - \texttt{fgets} which has a length parameter and cannot go over this length.

    \section{Accessing the Characters in a String}
    We could use the array subscripting or pointer arithmetic to process strings. Pointers simplify this. There is no difference between string parameter being declared as an array or as a pointer.

    \section{Using the C String Library}
    Header \texttt{string.h} includes many functions that are helpful for processing strings.

    \subsection*{Short Explanation of Some Functions}
    \begin{itemize}
    \item \texttt{char *strcpy(char s1, char s2)} - copies the string s2 (until first null character) into string s1 and
        returns s1. Function \texttt{strncpy} has a third argument which is the size of the string, is safer but slower.
    \item \texttt{size\_t strlen(const char *s)} - returns the length of the string (number of characters is up to,
        but not including, the first null character).
    \item \texttt{char *strcat(char *s1, const char *s2)} - appends the contents of the string s2 to the end of the string s1; it returns s1 (a pointer to the resulting string). There is also a \texttt{strncat} function.
    \item \texttt{int strcmp(const char *s1, const char *s2)} - compares the strings s1 and s2, returning a value less than, equal to, or greater than 0, depending on whether s1 is less than, equal to, or greater than s2.
    \end{itemize}

    \section{String Idioms}

    \subsubsection*{Searching for the End of a String:}
    \begin{lstlisting}
    while (*s++)
        ;
    \end{lstlisting}

    \subsubsection*{Copying a String:}
    \begin{lstlisting}
    while (*p++ = *s2++)
        ;
    \end{lstlisting}

    \section{Arrays of Strings}

    To fight the inefficiency in storing arrays of strings we need to use the \texttt{ragged array} - a two-dimensional array whose rows can have different lengths. This can be achived by creating the array of pointers to strings, where every element is a pointer to one string which could be of any size. Then these string can be accessed simply by the array's subscript.

    \subsection*{Command-Line Arguments}

    To obtain access to command-line arguments (called program parameters), we must define \texttt{main} as a function with two parameters:
    \begin{lstlisting}
    int main(int argc, char *argv[])
    {
        ...
    }
    \end{lstlisting}

    \texttt{argc} ("argument count") is the number of command-line arguments (including the name of the program itself). \texttt{argv} ("argument vector") is an array of pointers to the command-line arguments, which are stored in string form. \texttt{argv} has one additional element \texttt{argv[argc]} which is always a null pointer.

    

    \chapter{The Preprocessor}

    \texttt{\#define} and \texttt{\#include} (and any that begin with a \texttt{\#} character) directives are handled by the preprocessor, a piece of software that edits C programs just prior to compilation.

    \section{How the Preprocessor Works}
    
    The \texttt{\#define} directive defines a macro - a name that represents something else, e.g. \texttt{\#define WIDTH 20}. \\
    The \texttt{\#include} directive tells the preprocessor to open a particular file and "include" its contents as part of the file being compiled.

    \section{Preprocessing Directives}
    
    Most preprocessing directives fall into one of three categories:
    \begin{itemize}
        \item \textbf{Macro definition} - The \texttt{\#define} directive defines a macro; the \texttt{\#undef} directive removes a macro definition.
        \item \textbf{File inclusion} - The \texttt{\#include} directive causes the contents of a specified file to be included in a program.
        \item \textbf{Conditional compilation} - The \texttt{\#if, \#ifdef, \#ifndef, \#elif, \#else} and \texttt{\#endif} directives allow blocks of text to be either included in or exclude from a program, depending on conditions that can be tested by the preprocessor.
    \end{itemize}

    \bigskip

    Rules that apply to all directives:
    \begin{itemize}
        \item Directives always begin with the \texttt{\#} symbol.
        \item Any number of spaces and horizotal tab characters may separate the tokens in a directive.
        \item Directive always end at the first new-line character, unless explicitly continued (continue by adding a \textbackslash \phantom{ } character at the end of the line).
        \item Directives can appear anywhere in a program.
        \item Comments may appear on the same line as a directive.
    \end{itemize}

    \section{Macro Definitions}

    \subsection*{Simple Macros}

    Have the form: \texttt{\#define identifier replacement-list}. \texttt{replacement-list} is any sequence of preprocessing tokens.

    \subsection*{Parameterized Macros}

    Have the form: \texttt{\#define identifier( x1 , x2 , ... , xn ) replacement-list}. Where \texttt{x1, x2, ..., xn} are identifiers (the macro's parameters). The parameters may appear as many times as desired in the replacement-list.
    
    \bigskip

    For example. we've defined the following macro: \\
    \texttt{\#define MAX(x,y) ((x)>(y)?(x):(y))} \\
    And now we invoke it: \\
    \texttt{i = MAX(a, b);} \\
    What we get is: \\
    \texttt{i = ((a)>(b)?(a):(b))} \\
    Which works as a simple function.

    \subsection*{The \# Operator}

    The \# operator converts a macro argument into a string literal ("stringization"). We basically use it to print the inputted string into the outputted one.

    \bigskip

    \texttt{\#define PRINT\_INT(n) printf(\#n " = \%d\textbackslash n", n)} \\
    Invoking it: \\
    \texttt{PRINT\_INT(i);} \\
    Will create: \\
    \texttt{printf("i = \%d\textbackslash n", i);}

    \subsection*{The \#\# Operator}

    Can "paste" two tokens together to form a single token.

    \bigskip

    \texttt{\#define MK\_ID(n) i\#\#n} \\
    Invoking it: \\
    \texttt{MK\_ID(1)}\\
    The preprocessor will join i and 1 to make a single token (i1): \\
    \texttt{int MK\_ID(1);} \\
    After preprocessing, this declaration will become: \\
    \texttt{int i1;} \\
    
    \subsection*{General Properties of Macros}

    Are the following:
    \begin{itemize}
        \item A macro's replacement list may contain invocations of other macros.
        \item The preprocessor replaces only entire tokens, not portions of tokens.
        \item A macro definition normally remains in effect until the end of the file in which it appears.
        \item A macro may not be defined twice unless the new definition is identical to the old one.
        \item Macros may be "undefined" by the \#undef directive.
    \end{itemize}

    \section{Conditional Compilation}

    The inclusion or exclusion of a section of program text depending on the outcome of a test performed by the preprocessor.

    \subsection*{The \texttt{\#if} and \texttt{\#endif} Directives}

    For example:
    \begin{lstlisting}
    #define DEBUG 1
    ...
    #if DEBUG
    printf("Value of i: %d\n", i);
    printf("Value of i: %d\n", j);
    #endif
    \end{lstlisting}

    Will only leave out the \texttt{printf} calls when the DEBUG is 1. 

    \subsection*{The \texttt{defined} Operator}

    Produces the value 1 if the identifier is a currently defined macro; it produces 0 otherwise. E.g.
    \begin{lstlisting}
    #define DEBUG 1
    ...
    #if defined(DEBUG)      // or #if defined DEBUG
    ...
    #endif
    \end{lstlisting}
     
    \subsection*{The \texttt{\#ifdef} and \texttt{\#ifndef} Directives} 

    The \texttt{\#ifdef} directive tests whether an identifier is currently defined as a macro. E.g.
    \begin{lstlisting}
    #define DEBUG 1
    ...
    #ifdef DEBUG
    ...
    #endif
    \end{lstlisting}

    \bigskip

    The \texttt{\#ifndef} directive tests wheter an identifier is not defined as a macro.

    \subsection*{The \texttt{\#elif} and \texttt{\#else} Directives}

    For example:

    \begin{lstlisting}
    #if expr1
    ...
    #elif expr2
    ...
    #else
    ...
    #endif
    \end{lstlisting}
    
    The \texttt{\#ifdef} and \texttt{\#ifndef} directives could also be used.

    \subsection*{Uses of Conditional Compilation}
    \begin{itemize}
        \item Writing programs that are portable to several machines or operating systems.
        \item Writing programs that can be compiled with different compilers.
        \item Providing a default definition for a macro.
        \item Temporarily disabling code that contains comments.
    \end{itemize}

    \section{Miscellaneous Directives}

    \subsection*{The \texttt{\#error} Directive}

    If the preprocessor encounters an \texttt{\#error} directive, it prints an error message. Often used with conditional compilation. The message is any sequence of tokens, not a string literal. E.g.

    \begin{lstlisting}
    #if INT_MAX < 100000
    #error int type is too small
    #endif
    \end{lstlisting}
    Will produce: \texttt{Error directive: int type too small}
    
    \subsection*{The \texttt{\#line} Directive}
    Is used to alter the way program lines are numbered. Has two forms: \texttt{\#line n} and \texttt{\#line n "file"}. Used rarely.

    \subsection*{The \texttt{\#pragma} Directive}
    Provides a way to request special behavior from the compiler. Has the form: \texttt{\#pragma tokens}.

    \subsection*{The \texttt{\_Pragma} Operator}
    Has the form: \texttt{\_Pragma ( string literal )}



    \chapter{Writing Large Programs}

    \section{Source FIles}
    A C program can consists of many source files (parts of it with a .c extension). One of which must contain a \texttt{main} function.

    \section{Header Files}
    The \texttt{\#include} directive tells the preprocessor to open a specified file and insert its contents into the current file. The have the extension .h. In these header files we can include type definitions, function prototypes, and variable declarations so that other source files can see and use these.

    \subsection*{The \texttt{\#include} Directive}
    Has two forms, first is used for header files that belong to C's own library: \\
    \texttt{\#include <filename>} \\
    The second form is used for all other header files, including any that we write: \\
    \texttt{\#include "filename"}

    \subsection*{Sharing Variable Declarations}
    To declare i without defining it, we must put the keyword \texttt{extern} at the beginning of its declaration: \\
    \texttt{extern int a[];} \\
    We first put a definition of i in one file: \\
    \texttt{int i;} \\
    If i needs to be initialized, the initializer would go here. When this file is compiled, the compiler will allocate storage for i. The other files will contain declarations of i: \\
    \texttt{extern int i;} \\
    Because of the word \texttt{extern}, however the compiler does not allocate additional storage for i each time one of the files is compiled.

    \newpage
    \subsection*{Protecting Header Files}

    For example, we define a \texttt{boolean.h}:
    \begin{lstlisting}
    #ifndef BOOLEAN_H
    #define BOOLEAN_H

    #define TRUE 1
    #define FALSE 0
    typedef int Bool;

    #endif
    \end{lstlisting}

    \section{Building a Multiple-File Program}
    
    \subsection*{Makefiles}
    A file containing the information necessary to build a program. Also describes dependencies among the files. Are handy for compiling and linking multiple-file programs and partial compilation. For description of the \texttt{Makefile}, confront pages 367 - 368.

    \subsection*{Defining Macros Outside a Program}
    Most compilers (including GCC) support the -D option, which allows the value of a macro to be specified on the command line: \\
    \texttt{gcc -D DEBUG=1 foo.c}.



    \chapter{Structures, Unions, and Enumerations}

    \section{Structure Variables}
    The members of the structure aren't required to have the same type. To select a member we specify its name, not its position.
    
    \subsection*{Declaring Structure Variables}

    \begin{lstlisting}
    struct {
        int number;
        char name[NAME_LEN+1];
        int on_hand;
    } part1, part2;
    \end{lstlisting}

    There are two parts - two blocks of memory with all these variables. The members of a structure are stored in memory in order in which they are declared. Each structure represents a new scope; any names declared in that scope won't conflict with other names in a program. 

    \subsection*{Initializing Structure Variables}

    \begin{lstlisting}
    struct {
        int number;
        char name[NAME_LEN+1];
        int on_hand;
    } part1 = {528, "Disk Drive", 10},
      part2 = {914, "Printer Cable", 5};
    \end{lstlisting}

    The values in the initializer must appear in the same order as the members of the structure.

    \subsection*{Operations on Structures}

    \texttt{printf("Part number: \%d\textbackslash n", part1.number);} \\
    \texttt{part1.number = 258;} \\
    \texttt{scanf("\%d", \&part1.on\_hand);} \\
    \texttt{par2 = part1; \textbackslash \textbackslash copies every element from part1 to part2} \\
    
    \section{Structure Types}

    \subsection*{Declaring a Structure Tag}
    Is a name used to identify a particular kind of structure.

    \begin{lstlisting}
    struct part {
        int number;
        char name[NAME_LEN+1];
        int on_hand;
    };
    \end{lstlisting}

    Then we can use it to declare variables: \\
    \texttt{struct part part1, part2;}
    
    \bigskip

    But we could also combine them in declaration:
    \begin{lstlisting}
    struct part {
        int number;
        char name[NAME_LEN+1];
        int on_hand;
    } part1, part2;
    \end{lstlisting}

    Then they are compatible: \\
    \texttt{struct part part1 = \{528, "Disk Drive", 10\};} \\
    \texttt{par2 = part1;}

    \subsection*{Defining a Structure Type}

    \begin{lstlisting}
    typedef struct {
        int number;
        char name[NAME_LEN+1];
        int on_hand;
    } Part;
    \end{lstlisting}

    \texttt{Part part1, part2; \textbackslash \textbackslash not struct Part, because of typedef}

    \section{Nested Arrays and Structures}

    \subsection*{Nested Structures}
    In one structure we can declare an array, and then include this structure as a part of a larger structure. Then we would go in the reverse order (from the largest) to access an array member.

    \subsection*{Arrays of Structures}
    \texttt{struct part inventory[100];} \\
    Is capable of storing information about 100 parts. To access it: \\
    \texttt{print\_part(inventory[i]);} \\
    To assign 883 to the number member: \\
    \texttt{inventory[i].number = 883;} \\
    To access a singe character in a part name: \\
    \texttt{inventory[i].name[0] = '\textbackslash 0';} \\

    \section{Unions}
    The compiler allocates only enough space for the largest of the members, which overlay each other within this space. As a result, assigning a new value to one member alters the values of the other members as well.

    \begin{lstlisting}
    union {
        int i;
        double d;
    } u;

    \\ occupy most of the union (because int is less than double)
    u.i = 82;
    \\ alter i and now union stores only d
    u.d = 74.8;
    \end{lstlisting}

    \section{Enumerations} 
    An enumerated type is a type whose values are listed ("enumerated") by the programmer, who must create a name (an enumeration constant) for each of values.
    
    \subsection*{Enumeration Tags and Type Names}

    \texttt{enum suit \{CLUBS, DIAMONDS, HEARTS, SPADES\};} \\
    \texttt{enum suit s1, s2;}

    \subsection*{Enumerations as Integers}
    By the default enumeration variables are assigned to numbers 0, 1, 2, ... We could declare them in the other order: \\
    \texttt{enum suit \{CLUBS = 1, DIAMONDS = 2, HEARTS = 3, SPADES = \};} \\



    \chapter{Advanced Uses of Pointers}

    \section{Dynamic Storage Allocation}
    Dynamic storage allocation - the ability to allocate storage during program execution. We use it to design various data structures.

    \subsection*{Memory Allocation Functions}
    From \texttt{<stdlib.h>} header:
    \begin{itemize}
        \item \texttt{malloc} - Allocates block of memory but doesn't initialize it.
        \item \texttt{calloc} - Allocates block of memory and clears it.
        \item \texttt{realloc} - Resizes a previously allocated block of memory.
    \end{itemize}

    \subsection*{Null Pointers}
    Function returns a null pointer when cannot satify a request for memory allocation. E.g.
    \begin{lstlisting}
    if ((p = malloc(10000)) == NULL) {
        /* allocation failed; take appropriate action */
    }
    \end{lstlisting}

    \section{Dynamically Allocated Strings}

    \subsection*{Using \texttt{malloc} to Allocate Memory for a String}
    Which has the following prototype: \\
    \texttt{void *malloc(size\_t size);} \\
    Allocates a block of \texttt{size} bytes and returns a pointer to it.

    \bigskip

    \begin{lstlisting}
    p = malloc(n + 1);    // p is an empty string (one for the null char)
    strcpy(p, "abc");     // initialize it
    \end{lstlisting}

    \section{Dynamically Allocated Arrays}

    \subsection*{Using \texttt{malloc} to Allocate Storage for an Array}

    \begin{lstlisting}
    int *a;
    a = malloc(n * sizeof(int));
    \end{lstlisting}

    \subsection*{The \texttt{calloc} Function}
    \texttt{calloc} allocates space for an array with \texttt{nmemb} elements, each of which is \texttt{size} bytes long. Initializes all elements to 0.
    \begin{lstlisting}
    void *calloc(size_t nmemb, size_t size);

    a = calloc(n, sizeof(int));
    \end{lstlisting}

    \subsection*{The \texttt{realloc} Function}
    \texttt{realloc} resizes to a new \texttt{size} a block pointed to by the \texttt{ptr} pointer.
    \begin{lstlisting}
    void *realloc(void *ptr, size_t size);

    a = calloc(n, sizeof(int));
    \end{lstlisting}

    \section{Deallocating Storage}
    In order to collect garbage and prevent memory leaks, we need to use the \texttt{free} function.

    \subsection*{The \texttt{free} Function}

    \begin{lstlisting}
    void free(void *ptr);

    p = malloc(...);
    q = malloc(...);
    free(p);          // p won't be used anymore
    p = q;
    \end{lstlisting}
    Calling \texttt{free} releases the block of memory that p points to.

    \section{Linked Lists}
    Are the collection of memory allocated nodes, that point one to another, and end with the one containing the null pointer. Can grow and shrink if needed, but cannot be random accessed, meaning that we would need to go over every one and access them that way. For creating, editing, processing etc. on linked lists, read the \textbf{Linked Lists} section on pages 424 - 438.

    \subsection*{The \texttt{->} Operator}
    Right arrow selection is useful for accessing a member of a structure using a pointer. We can write: \\
    texttt{new\_node->value = 10;} \\
    instead of: \\
    \texttt{(*new\_node).value = 10;}

    \newpage

    \section{Pointers to Functions}
    Functions occupy memory locations.

    \begin{lstlisting}
    double integrate(double (*f)(double), double a, double b);
    \* we could also declare f as though it were a function *\
    double integrate(double f(double), double a, double b);

    result = integrate(sin, 0.0, PI / 2);
    \end{lstlisting}
    \texttt{sin} is without parentheses, because then compiler produces a pointer to the function instead of generating code for a function call.

    \section{Restricted Pointers (C99)}

    \texttt{int restrict p;}
    The keyword \texttt{restrict} means that p cannot be accessed in any way other than through p.

    \section{Flexible Array Members (C99)}

    When the last member of a structure is an array, its length may be ommited:

    \begin{lstlisting}
    struct vstring {
        int len;
        char chars[];   /* flexible array member - C99 only */
    }

    struct vstring *str = malloc(sizeof(struct vstring) + n);
    str->len = n;
    \end{lstlisting}

    The length of the \texttt{chars} array isn't determined until memory is allocated for a \texttt{vstring} structure. The \texttt{sizeof} operator ignores the \texttt{chars} member when computing the size of the structure. It takes up no space within a structure.



    \chapter{Declarations}

    \section{Declaration Syntax}

    Declaration has the following appearance: \\
    \texttt{declaration-specifiers declarators ;}

    \bigskip

    Declaration specifiers describe the properties of the variables or functions being delcared. Declarators give ther names and may provide additional information about their properties. Declaration specifiers fall into three categories:
    \begin{itemize}
        \item \textbf{Storage class} - \texttt{auto, static, extern, register}.
        \item \textbf{Type qualifiers} - \texttt{const, voatile, restrict}.
        \item \textbf{Type specifiers} - \texttt{void, char, short, int, long, float, double, signed, unsigned}.
    \end{itemize}

    \bigskip

    \texttt{static float x, y, *p;}

    \bigskip
    \texttt{static} - storage class \\
    \texttt{float} - type specifier \\
    \texttt{x, y, *p} - declarators \\

    \texttt{extern const unsigned long int a[10] = "foobar";}

    \bigskip
    \texttt{extern} - storage class \\
    \texttt{const} - type qualifier \\
    \texttt{unsigned long int} - type specifiers \\
    \texttt{a[10]} - declarator \\
    \texttt{"foobar"} - initializer \\

    \section{Storage Classes}

    \subsection*{Properties of Variables:}

    \begin{itemize}
        \item \textbf{Storage duration} - Determines when memory is set aside for the variable and when that memory is released. Can be automatic (automatically allocated and deallocated) and static (stays until program termination).
        \item \textbf{Scope} - The portion of the program text in which the variable can be referenced. Can have block scope (the variable is visible from its point of declaration to the end of the enclosing block) or file scope (whole file).
        \item \textbf{Linkage} - Determines the extent to which it can be shared by different parts of a program. External linkage means that it can be shared by several files in a program. But internal means only one. No linkage means that a variable belongs to a single function and can't be shared at all.
    \end{itemize}

    \begin{lstlisting}
    int i;

    void f(void)
    {
        int j;
    }
    \end{lstlisting}
    
    \texttt{i} has the static storage duration, file scope and external linkage. \texttt{j} has an automatic storage duration, block scope and no linkage.

    \subsection*{The \texttt{auto} Storage Class}
    The default for variables declared inside a block.

    \subsection*{The \texttt{static} Storage Class}
    Can be used everywhere. When used outside a block, a variable has internal linkage. When used inside a block, changes variable's storage duration from automatic to static.

    \subsection*{The \texttt{extern} Storage Class}
    Enables several source files to share the same variable. Always has a static storage duration. The scope of the variable depends on the declaration's placement.

    \subsection*{The \texttt{register} Storage Class}
    Asks the compiler to store the variable in a register instead of keeping it in main memory like other variables. Legal only in block. Everything as an \texttt{auto} variable.

    \subsection*{The Storage Class of a Function}
    \texttt{extern} - the function has external linkage. \texttt{static} - the function has internal linkage. If not specified function has external linkage.

    \section{Type Qualifiers}
    There are three type qualifiers: \texttt{const}, \texttt{volatile}, and \texttt{restrict} (C99). Const variables are only read-only. 

    \section{Declarators}
    \texttt{int i; \\
    int *ap[10]; \\
    float *fp(float); \\
    void (*pf)(int);}

    \subsection*{Deciphering Complex Declarations}
    \begin{itemize}
        \item Always read declarators from the inside out.
        \item When there's a choice, always favor \texttt{[]} and \texttt{()} over \texttt{*}.
    \end{itemize}

    \texttt{int *ap[10];} - array of pointers \\
    \texttt{float *fp(float);} - function that returs a pointer \\
    \texttt{void (*pf)(int);} - pointer to function with \texttt{int} argument returning \texttt{void} \\
    \texttt{int *(*x[10])(void);} - array of pointers to functions with no arguments returning pointer to \texttt{int}

    \section{Initializers}
    To initialize a variable, we write the \texttt{=} symbol after its declarator, then follow that with an initializer. E.g. \texttt{int i = 5 / 2;}.

    Some additional rulse that govern initializers:
    \begin{itemize}
        \item An initializer for a variable with static storage duration must be constant. E.g. \texttt{static int i = LAST - FIRST + 1;}.
        \item If a variable has automatic storage duration, its initializer need not be constant.
        \item A brace-enclosed initializer for an array, structure, or union must contain only constant expressions, never variables or function calls.
        \item The initializer for an automatic structure or union can be another structure or union.
    \end{itemize}

    \section{Inline Functions (C99)}
    By using the keyword \texttt{inline} we can improve performance of the program when calling a function many times.

    

    \chapter{Program Design}

    \section{Modules}
    Module - collection of services (functions), some of which are available to other parts of the program (the clients). Each module has an interface (header file) that describes the available services. The details of the module -- including the source code for the services themselves -- are stored in the module's implementation (source files).

    \subsection*{Types of Modules}
    \begin{itemize}
        \item \textbf{Data pool} - collection of related variables and/or constants.
        \item \textbf{Library} - collection of related functions.
        \item \textbf{Abstract object} - collection of functions that operate on a hidden data structure.
        \item \textbf{Abstract data type (ADT)} - a type whose representation is hidden.
    \end{itemize}

    \section{Information Hiding}
    Deliberately concealing information from the clients of a module. Is good for security and flexibility.

    \section{Abstract Data Types}
    We can create multiple abstract data types, and use information hiding to use them in clients. We can use incomplete types to hide objects inside our abstract data type. Incomplete types cannot be declared, but we can declare pointer to it.



    \chapter{Low-Level Programming}
    
    \section{Bitwise Operators}
    
    \subsection*{Bitwise Shift Operators}
    Can transform the binary representation of an integer by shifting its bits to the left or right. \\
    \texttt{<<} left shift \\
    \texttt{>>} right shift

    \texttt{i << j} is the result when the bits in \texttt{i} are shifted by \texttt{j} places.

    \subsection*{Bitwise Complement, And, Exclusive Or, and Inclusive Or}
    \texttt{\~} bitwise complement (unary) \\
    \texttt{\&} bitwise and (binary) \\
    \texttt{\^} bitwise exclusive or (binary) \\
    \texttt{|} bitwise inclusive or (binary) \\

    \subsection*{Using the Bitwise Operators to Access Bits}
    \begin{itemize}
        \item Setting a bit
        \item Clearing a bit
        \item Testing a bit
    \end{itemize}

    \subsection*{Using the Bitwise Operators to Access Bit-Fields}
    Bit-field - a group of several consecutive bits. Operations:
    \begin{itemize}
        \item Modifying a bit-field
        \item Retrieving a bit-field
    \end{itemize}

    \section{Other Low-Level Techniques}
    
    \subsection*{The \texttt{volatile} Type Qualifier}
    The \texttt{volatile} type qualifier allows us to inform the compiler if any of the data in a program is volatile, meaning that it can change its address as a proram is running. This is used mostly for declarations of a pointer variables taht will point to a volatile memory location e.g. most recent character typed at the user's keyboard.



    \chapter{The Standard Library}

    \section{Using the Library}

    \subsection*{Restrictions on Names Used in the Library}
    \begin{itemize}
        \item Identifiers that begin with an underscore followed by an upper-case letter or a second underscore are reseved for use within the library.
        \item Identifiers that begin with an underscore are reserved for use as identifiers and tags with file scope.
        \item Every identifier with external linkage in the standard library is reserved for use as an identifier with external linkage.
    \end{itemize}

    \section{C89 Library Overview}
    \texttt{assert.h} - Diagnostics \\
    \texttt{ctype.h} - Character Handling \\
    \texttt{errno.h} - Errors \\
    \texttt{float.h} - characteristics of Floating Types \\
    \texttt{limits.h} - Sizes of Integer Types \\
    \texttt{locale.h} - Localization \\
    \texttt{math.h} - Mathematics \\
    \texttt{setjmp.h} - Nonlocal Jumps \\
    \texttt{signal.h} - Signal Handlin \\
    \texttt{stdarg.h} - Variable Arguments \\
    \texttt{stddef.h} - Common Definitions \\
    \texttt{stdio.h} - Input/Output \\
    \texttt{stdlib.h} - General Utilities \\
    \texttt{string.h} - String Handling \\
    \texttt{time.h} - Date and Time \\

    \section{C99 Library Changes}
    \begin{itemize}
        \item Additional headers
        \item Additional Macros and functions
        \item Enhanced versions of existing functions
    \end{itemize}

    \texttt{complex.h} - Complex Arithmetic \\
    \texttt{fenv.h} - Floating-Point Environment \\
    \texttt{inttypes.h} - Format Conversion of Integer Types \\
    \texttt{iso646.h} - Alternative Spelling \\
    \texttt{stdbool.h} - Boolean Type and Values \\
    \texttt{stdint.h} - Integer Types \\
    \texttt{tgmath.h} - Type-Generic Math \\
    \texttt{wchar.h} - Extended Multibyte and Wide-Character Utilities \\
    \texttt{wctype.h} - Wide-Character Classification and Mapping Utilities \\

    


    \chapter{Input/Output}
    
    \section{Streams}
    \textbf{Stream} means any source of input or any destination for output.

    \subsection*{File Pointers}
    Accessing a stream in a C program is done through a file pointer e.g. \\
    \texttt{FILE *fp1, *fp2;}

    \subsection*{Standard Streams and Redirection}
    They can't be declared, opened or closed. 
    \begin{itemize}
        \item \texttt{stdin} - Standard Input
        \item \texttt{stdout} - Standard Output
        \item \texttt{stderr} - Standard Error
    \end{itemize}
    
    We could redirect an output or input from these to other streams.

    \subsection*{Text Files versus Binary Files}
    The bytes in a text file represent characters, making it possible for a human to examine the file or edit it. In a binary file, bytes don't necessarily represent characters; groups of bytes might represent other types of data.

    \bigskip

    Text files are divided into lines, and may contain a special "end-of-file" marker. Binary files aren't divided into lines. In a binary file, there are no end-of-lile or end-of-file markers.

    \section{File Operations}

    \subsection*{Opening a File}
    \texttt{FILE *fopen(const char * restrict filename, const char * restrict mode);}

    \bigskip

    The second argument is a "mode string" that specifies what operations we intend to perform on the file. The word \texttt{restrict} (a C99 keyword), indicates that \texttt{filename} and \texttt{mode} should point to strings that don't share memory locations.

    \smallskip
    
    \texttt{fp = fopen("in.dat", "r");   /* opens in.dat for reading */}

    \smallskip

    When it can't open a file, it returns a null pointer.

    \newpage

    \bigskip

    \begin{table}
        \centering
        % align the left column to the left
        \begin{tabular}{|c|>{\raggedright\arraybackslash}p{8cm}|}
        \hline
        String & Meaning \\
        \hline
        "r" & Open for reading \\
        \hline
        "w" & Open for writing (file need not exist) \\
        \hline
        "a" & Open for appending (file need not exist) \\
        \hline
        "r+" & Open for reading and writing, starting at beginning \\
        \hline
        "w+" & Open for reading and writing (truncate if file exists) \\
        \hline
        "a+" & Open for reading and writing (append if file exists) \\
        \hline
        \end{tabular}
        \caption{Text Modes}
        \label{tab:example}
    \end{table}

    \begin{table}
        \centering
        % align the left column to the left
        \begin{tabular}{|c|>{\raggedright\arraybackslash}p{8cm}|}
        \hline
        String & Meaning \\
        \hline
        "rb" & Open for reading \\
        \hline
        "wb" & Open for writing (file need not exist) \\
        \hline
        "ab" & Open for appending (file need not exist) \\
        \hline
        "rb+" & Open for reading and writing, starting at beginning \\
        \hline
        "wb+" & Open for reading and writing (truncate if file exists) \\
        \hline
        "ab+" & Open for reading and writing (append if file exists) \\
        \hline
        \end{tabular}
        \caption{Binary Modes}
        \label{tab:example}
    \end{table}

    \subsection*{Closing a file}

    \texttt{int fclose(FILE *stream);}

    \bigskip

    Closes a file, the argument must be a \texttt{FILE *} pointer to it. Returns zero if the file was closed successfully; otherwise, it returns error code EOF.

    \subsection*{Attaching a File to an Open Stream}

    \texttt{FILE *freopen(const char * restrict filename, const char * restrict mode, FILE * restrict stream);}

    \bigskip

    Mostly is used to associate a file with one of the standard streams.

    \subsection*{Temporary Files}

    \texttt{FILE *tmpfile(void);} \\
    \texttt{char *tmpnam(char *s);} \\

    \bigskip

    \texttt{tmpfile} creates a temporary file that only exists as long as the program is running. \texttt{tmpnam} is used to create a temporary file name.

    \subsection*{File Buffering}

    \begin{lstlisting}
    int fflush(FILE *stream);
    void setbuf(FILE * restrict stream, char * restrict buf);
    int setvbuf(FILE * restrict stream, char * restrict buf, int mode, size_t size);
    \end{lstlisting}

    \texttt{fflush(fp);} flushes buffer for fp. \\
    \texttt{fflush(NULL);} flushes all buffers.

    \texttt{setvbuf} allows us to change the way a stream is buffered and to control the size and location of the buffer.

    \subsection*{Miscellaneous File Operations}

    \begin{lstlisting}
    int remove(const char *filename);
    int rename(const char *old, const char *new);
    \end{lstlisting}

    They work with file names instead of file pointers.

    \smallskip

    \texttt{remove("foo"); /* deletes the file named "foo" */} \\
    \texttt{rename("foo", "bar"); /* renames "foo" to "bar" */}

    \smallskip

    Be sure that the files has been closed. \\

    \section{Formatted I/O}

    So every tex compilator displays an error for this file, but worked 15 minutes ago on first try. Won't update it because I got a lot more interesting projects to do and this will take too much time.

\end{document}

\documentclass[12pt, openany]{book}

\usepackage[T1]{fontenc}
\usepackage[utf8]{inputenc}
\usepackage{listings, lstautogobble}        % code segments
\usepackage{array, multirow}                % tables multirows
\usepackage[table]{xcolor}                  % tables colors

% inverts colors of the document, I use it mostly for my eyes' sake in night
\pagecolor[rgb]{0.2, 0.2, 0.2}
\color[rgb]{1, 1, 1}

% set code segment's parameters
\lstset{language=C,
    basicstyle=\small\ttfamily,
    stringstyle=\ttfamily,
    showstringspaces=false,
    autogobble=true
}

\newcommand\tab[1][20px]{\hspace*{#1}}      % tabs 20px
\setlength{\parindent}{0pt}                 % no tabs at start of the paragraphs

\title{My notes from K.N. King's "C Programming A Modern Approach" 2nd version}
\author{Piotr Marendowski}
\date{March 2023}

\begin{document}
    \maketitle

    \chapter{Note}
    In this material I will go over everything from book,
    trying to summarize every note-worthy subject. I will do
    it, while learning Latex, so good luck to me.

    \tableofcontents

    % PAGE 1
    \chapter{C Fundamentals}
    
    \section{Steps of Executing a C Program}
    Automated proccess:
    \begin{enumerate}
        \item\textbf{Preprocessing} - Preprocessor is executing directives
        (they begin with \#).
        \item\textbf{Compiling} - Compiler translates program into machine
        instructions (object code).
        \item\textbf{Linking} - Linker combines object code and code needed
        for execution of the program.
    \end{enumerate}

    \section{The Genereral Form of a Simple Program}
    Simple C programs have this form:

    \bigskip
    \textit{directives}
    \smallskip

    \small\texttt{int main(void)} \\
    \small\texttt{\{} \\
    \tab\textit{statements} \\
    \small\texttt{\}}
    \bigskip

    \textbf{Directives} - Begin with '\#' symbol, they state what headers
    include to program.

    % PAGE 2
    \textbf{Functions} - They are segments of code that take arguments, and
    returns (or not) a value. Only \texttt{main} function is required.

    \textbf{Statements} - Commands to execute, mostly end with semicolon.

    \bigskip
    \textbf{String literal} - Series of characters enclosed in double quotation marks,
    e.g. \texttt{"Hello world!"}.

    \textbf{New-line character} - \texttt{\textbackslash n} is an escape sequence, which
    advances to the next line of output.

    \textbf{Comments} - Are ommited in program execution, can be used to comment single line
    e.g. \texttt{/* Comment */}, or block of lines. From C99 we can use one line comments
    e.g. \texttt{// Comment}.

    \section{Variables and Assigments}

    \textbf{Variable} - Place to store calculation's output, for using in future.
    Variable's characteristics:
    \begin{itemize}
        \item 
        \textbf{Types} - For now, there are two types of variables:
        \begin{itemize}
            \item \texttt{int} - Integer types, can store quite big whole number, but
            that depends on your computer's architecture.
            \item \texttt{float} - Can store bigger numbers, as well as digits after
            decimal point.
        \end{itemize}

        \item 
        \textbf{Declarations} - To use a variable, we first need to declare it. It
        means that we need to specify variable's type, and name. We can chain declarations
        with the same type e.g. \texttt{int i, sum, x;}. In C99 they can now be declared
        after statements, not like in C89.
        
        \item
        \textbf{Assignment} - We assign value to a variable. Variable is on the left
        side, while value, expression, formula etc. is on the right side. To assign
        something to a variable, we first need to declare it. Examples:
        \begin{lstlisting}
        int i;
        float f;
        i = 1;
        f = 1.5;
        \end{lstlisting}
    \end{itemize}

    \subsection*{Initialization}
    At the default most variables are uninitialized, which means that they have some
    random - garbage value assigned to them, if we didn't. In declaration we can assign
    value to a variable, making it an \textbf{initializer}, e.g. \texttt{int i = 0;}.

    % PAGE 3
    \section{Reading Input}
    For reading input we need to use \texttt{scanf} function, which needs a format
    string and value to read, e.g. \texttt{scanf("\%d", \&i);}.

    \section{Defining Names for Constants}
    To define a constant, we need to use a \textbf{macro definition}, which is
    interpreted by the preprocessor e.g. \texttt{\#define WIDTH 20}.

    \section{Identifiers}
    Names in C are called \textbf{identifiers}. They can begin with the lower-case or
    upper-case letters or underscores e.g. \texttt{times10  my\_var  \_done}. They
    cannot begin with a number e.g. \texttt{10times}. They cannot contain minus signs
    e.g. \texttt{my-var}.

    \subsection*{Keywords}
    There are number of keywords, which are prohibited from using as identifiers.

    \section{Layout of the C Program}
    We can slice C statements into \textbf{tokens}:
    \begin{lstlisting}
    printf   (   "Height:   %d\n"   ,   height   )   ;
      1      2      3         4     5     6      7   8
    \end{lstlisting}
    Tokens 1 and 2 are identifiers, token 3 is a string literal and tokens 2, 4, 6,
     and 7 are punctuation. \\
    In most cases we can put many spaces between them. But we cannot put spaces within
    tokens e.g. \texttt{fl oat f;}. 

    % PAGE 4
    \chapter{Formatted Input/Output}

    \section{The \texttt{printf} Function}
    Needs a format string and arguments to insert there. There is no limit to these
    arguments. Format string could have conversion specifications, which are supplied by
    its arguments, to insert into format string e.g. \texttt{printf("Value: \%d", i);}
    where \texttt{\%d} is a conversion specification and \texttt{i} is a value to be
    supplied to the format string.

    \section{Conversion Specifications}
    Outside of letter(s) specifying which type to covert to, they consist of the \textbf{Minimal
    Field Width (m)} and \textbf{Precision (p)}. They have the form of: \texttt{\%m.pX}.

    \subsection*{Minimal field width}
    Specifies the minimum number of characters to print. If the number of characters
    to print is less than specified, the number is right justified with spaces added. If the number of
    characters is greater than specified it will automatically expand to display all of characters.

    \subsection*{Precision}
    Depends on the type to be displayed, reference the book for more detailed preview. 
    \newpage

    % PAGE 5
    Conversion specifications:
    \begin{itemize}
        \item \texttt{d} - Displays an integer in a decimal (base 10) form. \textit{p} indicates
        the minimum number of digits to display.
        \item \texttt{e} - Displays a floating-point number in exponential format. \textit{p}
        indicates the number of digits after the decimal point.
        \item \texttt{f} - Displays a floating-point number without an exponent. \textit{p} has
        the same meaning as previous.
        \item \texttt{g} - Displays a floating-point number in exponential format or
        fixed (without an exponent). \textit{p} indicates the maximum number of
        significant digits to be displayed. It depends on the size of the number.
    \end{itemize}

    \subsection*{Escape sequences}
    They are characters, that would introduce problems in compilation or have some action
    to do e.g. insert new line into output. Few of them are:
    \begin{itemize}
        \item \texttt{\textbackslash a} - alert (bell),
        \item \texttt{\textbackslash b} - backspace,
        \item \texttt{\textbackslash n} - new line,
        \item \texttt{\textbackslash t} - tab,
        \item \texttt{\textbackslash "} - qoute character,
        \item \texttt{\textbackslash\textbackslash} - slash character.
    \end{itemize}

    \section{The \texttt{scanf} Function}
    This funcion handles input from stdin stream (keyboard),  have a format string
    and may contain conversion specifications. The scanf call may look like that:
    \texttt{scanf("\%d", \&i);}. Scanf when reading an input ignores the white-space
    characters. It only matches input to the provided variables. If a reading error
    occurres, scanf will return immediately, ignoring the rest of the format string.
    It doesn't read the new-line character at the end of the input. If character
    cannot be read, function puts it back for the next variable and adds it to that.

    % PAGE 6
    \subsection*{Ordinary Characters in Format Strings}
    We can put a white-space characters into the format string, then scanf will read
    any number of white-space characters and discard them. When it encounters a
    non-white-space character it is trying to match it with an inputted character.
    If it fails, it returns.

    % PAGE 7
    \chapter{Expressions}

    \section{Arithmetic Operators}
    \begin{tabular}{ |c|c|c| } 
        \hline
        \cellcolor{lightgray}Unary & \multicolumn{2}{|c|}{\cellcolor{lightgray}Binary} \\
        \hline
        & \cellcolor{lightgray}Additive & \cellcolor{lightgray}Multiplicative \\
        \cline{2-3}
        {}{}{\shortstack{+ unary plus \\ - unary minus}} & + addition & * multiplication \\ 
         & - subtraction & / division \\ 
         & & \% reminder \\
        \hline
    \end{tabular}

    \subsection*{Operator Precendence and Associativity}
    \textbf{Operator precendece} is in what order C calculates expressions. The arithmetic
    operators have the following relative precendence: \\
    \tab Highest: + - (unary) \\
    \tab\tab\tab \ * \ / \ \% \\    % single backslashes are spaces
    \tab Lowest: \ + - (binary)

    \bigskip
    \textbf{Associativity} decides in what order operators with the same precendece are
    calculated. The binary operators are all left associative, whilst the unary operators
    are all right associative.

    \section{Assignment Operators}
    Are used to store a computed value of the expression.

    % PAGE 8
    \subsection*{Simple Assignment}
    Evaluates an expression, which then assigns into the variable, expression can be a
    \textit{constant} (always has the same value). If they don't have the same type, the
    value of an expression is converted to the type of the variable. Assignments can be
    chained together e.g. \texttt{i = j = k = 0;}. The \texttt{=} operator is right
    associative. 

    \subsection*{Lvalues}
    Assignment operator requires that on its left side can only be a variable, not an
    expression e.g. \texttt{i + j = 0;} is wrong.

    \subsection*{Compound Assignment}
    We can shorten statements e.g. \texttt{i += 2;} is equivalent to
    \texttt{i = i + 2;}. It works with the other operators including the following:
    \texttt{-= *= /= \%=}, they all work in the same way and are right associative.

    \section{Increment and Decrement Operators}
    Used to even more shorten a compound addition and subtraction by 1. E.g. 
    \begin{lstlisting}
        i = i + 1;
        j = j - 1;
    \end{lstlisting}
    Are the same as:
    \begin{lstlisting}
        i += 1;
        j -= 1;
    \end{lstlisting}
    Which are the same as:
    \begin{lstlisting}
        i++;
        j--;
    \end{lstlisting}

    % it seems that \texttt need one more minus, don't know why
    They contain side effects - after adding or subtracting 1, value of their operands
    is modified. There are two types of these operators: \textbf{prefix} (\texttt{++i}
    or \texttt{---i}) which increments the variable first, then assigns value to \texttt{i},
    and \textbf{postfix} (\texttt{i++} or \texttt{i---}) which first assings value to
    \texttt{i} and increments \texttt{i} after this statement.

    % PAGE 9
    \chapter{Selection Statements}
    We could group most statements in this three categories:
    \begin{itemize}
        \item \textbf{Selection statements} - Test provided condition, and execute code
        within their borders, e.g. \texttt{if} and \texttt{switch} statements.
        \item \textbf{Iteration statements} - Iterate over and over again, until their
        condition is not true, e.g. \texttt{for, while}, and \texttt{do while} statements.
        \item \textbf{Jump statements} - They control the flow of the program, can stop
        iterations, skip through them or jump to any place in the program, e.g. \texttt{break},
        \texttt{continue} and \texttt{goto} statements.
    \end{itemize} 

    \section{Logical Expressions}
    \subsection*{Relational Operators}
    They are used to compare expressions, yelding 0 if statement are not true and 1 if
    statement are true.

    \bigskip
    \begin{tabular}{|@{} c l|}  % will do the trick :)
        \hline
        \textbf{ Symbol} & \textbf{Meaning} \\
        \hline
        < & less than \\
        > & greater than \\
        <= & less than or equal to \\
        >= & greater than or equal to \\
        \hline
    \end{tabular} 

    % PAGE 10
    \subsection*{Equality Operators}

    \bigskip
    \begin{tabular}{|@{} c l|}
        \hline
        \textbf{ Symbol} & \textbf{Meaning} \\
        \hline
        == & equal to \\
        != & not equal to \\
        \hline
    \end{tabular}

    \subsection*{Logical Operators}

    \bigskip
    \begin{tabular}{|@{} c l l|}
        \hline
        \textbf{ Symbol} & \textbf{Meaning} & \textbf{Operation} \\
        \hline
        ! & logical negation & Inverse - if false, returns 1\\
        \&\& & logical end & If both expressions are true, returns 1 \\
        || & logical or & If either one of them them is true, returns 1 \\ 
        \hline
    \end{tabular}

    \section{The \texttt{If} Statement}
    Has the form of: \texttt{if ( expression ) statement}. If evaluated expression has a
    non-zero value, then statement after parentheses is executed.

    \subsection*{Compound Statements}
    We can "stack" multiple statements between the parentheses.

    \subsection*{The \texttt{else} clause}
    If we want to execute statements when our expression is not true, we need to use
    a \texttt{else} clause. It has the following form: \texttt{if ( expression )
    statement else statement}.

    \subsection*{Cascaded \texttt{if} Statements}
    Thanks to them we can test multiple conditions. They have the following form:

    \texttt{if ( expression ) statement else if} \texttt{( expression ) statement else} \\
    \texttt{statement}.

    \subsection*{Conditional Expressions}
    Has the following form: \texttt{epr1 ? epr2 : epr3}. Tests whether the first
    expression is true, if it is then executes the second expression, otherwise
    executes the third expression.

    % PAGE 11
    \subsection{Boolean Values}
    \subsubsection{C89}
    There is not a boolean type in C89, but we could declare a macro definition named
    \texttt{TRUE or FALSE}, e.g. \texttt{\#define TRUE 1}.

    \subsubsection*{C99}
    With the arrival of C99 we could declare \texttt{\_Bool} type, e.g.
    \texttt{\_Bool flag = true;}. For using this type we must declare a
    \texttt{\#include <stdbool.h>}.
    
    \section{The \texttt{switch} statement}
    Switch statement can be a better-looking and faster alternative to the cascaded
    \texttt{if}. It has the following form:
    \begin{lstlisting}
        switch ( expression ) {
            case constant-expression : statements
            ...
            case constant-expression : statements
            default : statements
        }
    \end{lstlisting}
    
    Components of the \texttt{switch} statement:
    \begin{itemize}
        \item \textbf{Controlling expression} - could be an \texttt{int} or a
        \texttt{char}, but not \texttt{float} and strings.
        \item \textbf{Case lebels} - Form: \texttt{case constant-expression :}
        Cannot contain variables and function calls.
        \item \textbf{Statements} - Comes after each case label.
    \end{itemize}
    \texttt{Switch} statement doesn't need a \texttt{default} case. We use a
    \texttt{break} statement to stop at one choice, otherwise switch would execute
    all remaining cases.

    % PAGE 12
    \chapter{Loops}
    Loop is a repeatedly executing statements until the controlling expression
    is not true.
    
    \section{The \texttt{while} Statement}
    It has the following form: \texttt{while ( expression ) statement}. First
    it tests the controlling expression, then executes the loop body, if the
    expression is false, the loop terminates. It is possible that the loop
    body will not be executed at all, because it could be a false from the
    beginning.
    
    \section{The \texttt{do} Statement}
    It resembles the \texttt{while} loop, but it tests the controlling
    expression after each execution of the loop body, which makes it execute
    always at least once. It has the following form: \texttt{do statement
    while ( expression ) ; }.
    
    \section{The \texttt{for} Statement}
    Is used in a variety of ways, is ideal for counting loops. Has the following
    form: \texttt{for ( expr1 ; expr2 ; expr3 ) statement}. Every expression has
    its own role, in first you initialize or assign values to iterating variables.
    In the second you place a condition, which is checked every execution and while
    it is true loop will execute. In the third you put an operation which is performed
    at the end of each loop iteration. We could make more expressions (separated by
    the comma), but would need to place them in place according to their characteristics.
    Most of the \texttt{for} loops can be replaced by the \texttt{while} loop. We could
    omit any expression, but we need to compensate for it before or later. From C99
    we could declare variables in the loop body, which will not be used as the
    variables already declared.
    

    % PAGE 13
    \section{Exiting from a Loop}

    \subsection{The \texttt{break} Statement}
    Is used to jump out of the loop body, terminating it. Can escape only one level
    of nesting.

    \subsection{The \texttt{continue} Statement}
    Only used in loops. Transfers control to the end of the loop, but it stayes in
    loop for the next execution. Can be considered a
    "skip to the end" statement.

    \subsection*{The \texttt{goto} Statement}
    Jumps (transfers control) from its declaration to the label in the other part
    of the program. Label has the following form: \texttt{identifier : statement}.
    The \texttt{goto} statement has this form: \texttt{goto identifier ;}. Nowadays
    it is not commonly used, in favour of the \texttt{break, continue} and
    \texttt{return} statements.

    \section{The \texttt{null} Statement}
    We could omit the loop body (moving the loop body into expressions) or one of
    its expressions (which will create an infinite loop), e.g.
    \texttt{for (i = 0; ; i++) {...} }.

    % PAGE 14
    \chapter{Basic Types}

    \section{Integer Types}
    They are the whole numbers. We can divide them into two categories: \texttt{signed}
    and \texttt{unsigned}. The \texttt{signed} integers can be both positive and
    negative, but because of this they are two times less than \texttt{unsigned} type,
    which cannot contain negative numbers, thus making it one bit bigger. Sometimes we
    need bigger numbers, then we would use a \texttt{long} integer which on the most
    machines is double the \texttt{int} type. We could also store smaller integer values
    in the \texttt{short} type. Every type can be \texttt{signed} or \texttt{unsigned},
    e.g. \texttt{unsigned short int}. We could abbreviate names by dropping the word
    \texttt{int}. Types aren't required to have a certain bounds or maximum values, which
    varies from one architecture to another. For our machine's ranges we could see the
    \texttt{<limits.h>} header.

    \subsection*{Integer Types in C99}
    From C99 we could now declare type \texttt{long long int}, which can be even two
    times the \texttt{long int} type.

    \subsection*{Integer Constants}
    Constant numbers are numbers that cannot change. Can be of the base 10, 8 and 16:
    \begin{itemize}
        \item \textbf{Decimal} - Contain digits 0 through 9, cannot begin with a zero.
        \item \textbf{Octal} - Contain digits 0 through 7, must begin with a zero.
        \item \textbf{Hexadecimal} - Containg digits 0 through 9 and letter a to f,
        always begin with a 0x. Letters can be either lower case or upper case.
    \end{itemize}

    % PAGE 15
    We could mix together any of these without any repercussions. As well as we could
    indicate that constant is for example \texttt{long int} - by adding \texttt{L or l}
    to the number. To indicate that constant is a \texttt{unsigned int} -
    add \texttt{U or u}. Both can be added together. In C99 we could specify the type
    \texttt{long long int} by adding \texttt{LL or ll} to the end.

    \bigskip
    If the result of the arithmetic operation is more than what desired type can store,
    we say that occured an \textbf{Integer Overflow}.

    \subsection*{Reading and Writing Integers}
    For reading/writing operations we need to use the conversion specification
    for Unsigned types:
    \begin{itemize}
        \item \textbf{Unsigned} - \texttt{U} in decimal.
        \item \textbf{Octal} - \texttt{O}.
        \item \textbf{Hexadecimal} - \texttt{X}.
    \end{itemize} 
    We use could use them with type conversion specifications by appending this letter
    to the front:
    \begin{itemize}
        \item \textbf{Short} - \texttt{H}.
        \item \textbf{Long} - \texttt{L}.
        \item \textbf{Long Long} (C99) - \texttt{LL}.
    \end{itemize}

    \section{Floating Types}
    Numbers in the decimal form. There are three of them:
    \begin{itemize}
        \item \textbf{Float} - Single precision.
        \item \textbf{Double} - Double precision.
        \item \textbf{Long double} - Extended precision.
    \end{itemize}
    Mostly are stored according to The IEEE Floating-Point Standard.

    % PAGE 16
    \subsection*{Floating Constants}
    A floating constant must contain a decimal point and/or an exponent. The exponent
    (if present) must be preceded by the letter E or e. We could put the letter F or f
    (for \texttt{float}) or LF or lf (for \texttt{double}) at the end of the number to
    set the desired type.

\end{document}
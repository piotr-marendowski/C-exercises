\documentclass[12pt, openany]{book}

\usepackage[T1]{fontenc}
\usepackage[utf8]{inputenc}
\usepackage{listings}                       % code 
\usepackage{lstautogobble}                  % segments

% set code segment's parameters
\lstset{language=C,
    basicstyle=\small\ttfamily,
    stringstyle=\ttfamily,
    showstringspaces=false,
    autogobble=true
}

\newcommand\tab[1][20px]{\hspace*{#1}}      % tabs 20px
\setlength{\parindent}{0pt}                 % no tabs at start of the paragraphs

\title{My notes from K.N. King's "C Programming A Modern Approach" 2nd version}
\author{Piotr Marendowski}
\date{March 2023}

\begin{document}
    % PAGE 1
    \maketitle

    % PAGE 2
    \chapter{Note}
    In this material I will go over everything from book,
    trying to summarize every note-worthy subject. I will do
    it, while learning Latex, so good luck to me.

    % PAGE 3
    \tableofcontents

    % PAGE 4
    \chapter{C Fundamentals}
    
    \section{Steps of Executing a C Program}
    Automated proccess:
    \begin{enumerate}
        \item\textbf{Preprocessing} - Preprocessor is executing directives
        (they begin with \#).
        \item\textbf{Compiling} - Compiler translates program into machine
        instructions (object code).
        \item\textbf{Linking} - Linker combines object code and code needed
        for execution of the program.
    \end{enumerate}

    \section{The Genereral Form of a Simple Program}
    Simple C programs have this form:

    \bigskip
    \textit{directives}
    \smallskip

    \small\texttt{int main(void)} \\
    \small\texttt{\{} \\
    \tab\textit{statements} \\
    \small\texttt{\}}
    \bigskip

    \textbf{Directives} - Begin with '\#' symbol, they state what headers
    include to program.

    % PAGE 5
    \textbf{Functions} - They are segments of code that take arguments, and
    returns (or not) a value. Only \texttt{main} function is required.

    \textbf{Statements} - Commands to execute, mostly end with semicolon.

    \bigskip
    \textbf{String literal} - Series of characters enclosed in double quotation marks,
    e.g. \texttt{"Hello world!"}.

    \textbf{New-line character} - \texttt{\textbackslash n} is an escape sequence, which
    advances to the next line of output.

    \textbf{Comments} - Are ommited in program execution, can be used to comment single line
    e.g. \texttt{/* Comment */}, or block of lines. From C99 we can use one line comments
    e.g. \texttt{// Comment}.

    \section{Variables and Assigments}

    \textbf{Variable} - Place to store calculation's output, for using in future.
    Variable's characteristics:
    \begin{itemize}
        \item 
        \textbf{Types} - For now, there are two types of variables:
        \begin{itemize}
            \item \texttt{int} - Integer types, can store quite big whole number, but
            that depends on your computer's architecture.
            \item \texttt{float} - Can store bigger numbers, as well as digits after
            decimal point.
        \end{itemize}

        \item 
        \textbf{Declarations} - To use a variable, we first need to declare it. It
        means that we need to specify variable's type, and name. We can chain declarations
        with the same type e.g. \texttt{int i, sum, x;}. In C99 they can now be declared
        after statements, not like in C89.
        
        \item
        \textbf{Assignment} - We assign value to a variable. Variable is on the left
        side, while value, expression, formula etc. is on the right side. To assign
        something to a variable, we first need to declare it. Examples:
        \begin{lstlisting}
        int i;
        float f;
        i = 1;
        f = 1.5;
        \end{lstlisting}
    \end{itemize}

    \subsection*{Initialization}
    At the default most variables are uninitialized, which means that they have some
    random - garbage value assigned to them, if we didn't. In declaration we can assign
    value to a variable, making it an \textbf{initializer}, e.g. \texttt{int i = 0;}.

    % PAGE 6
    \section{Reading Input}
    For reading input we need to use \texttt{scanf} function, which needs a format
    string and value to read, e.g. \texttt{scanf("\%d", \&i);}.

    \section{Defining Names for Constants}
    To define a constant, we need to use a \textbf{macro definition}, which is
    interpreted by the preprocessor e.g. \texttt{\#define WIDTH 20}.

    \section{Identifiers}
    Names in C are called \textbf{identifiers}. They can begin with the lower-case or
    upper-case letters or underscores e.g. \texttt{times10  my\_var  \_done}. They
    cannot begin with a number e.g. \texttt{10times}. They cannot contain minus signs
    e.g. \texttt{my-var}.

    \subsection*{Keywords}
    There are number of keywords, which are prohibited from using as identifiers.

    \section{Layout of the C Program}
    We can slice C statements into \textbf{tokens}:
    \begin{lstlisting}
    printf   (   "Height:  \%d\n"   ,   height   )   ;
      1      2      3        4      5     6      7   8
    \end{lstlisting}
    Tokens 1 and 2 are identifiers, token 3 is a string literal and tokens 2, 4, 6,
     and 7 are punctuation. \\
    In most cases we can put many spaces between them. But we cannot put spaces within
    tokens e.g. \texttt{fl oat f;}. 


\end{document}
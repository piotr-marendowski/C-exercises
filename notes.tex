\documentclass[12pt, openany]{book}
\usepackage[T1]{fontenc}
\usepackage[utf8]{inputenc}
\newcommand\tab[1][1cm]{\hspace*{#1}}

\title{My notes from K.N. King's "C Programming A Modern Approach" 2nd version}
\author{Piotr Marendowski}
\date{March 2023}

\begin{document}
    % PAGE 1
    \maketitle

    % PAGE 2
    \chapter{Note}
    In this material I will go over everything from book,
    trying to summarize every note-worthy subject. I will do
    it, while learning Latex, so good luck to me.

    % PAGE 3
    \tableofcontents

    % PAGE 3
    \chapter{C Fundamentals}
    
    \section{Steps of Executing a C Program}
    Automated proccess:
    \begin{enumerate}
        \item\textbf{Preprocessing} - Preprocessor is executing directives
        (they begin with \#).
        \item\textbf{Compiling} - Compiler translates program into machine
        instructions (object code).
        \item\textbf{Linking} - Linker combines object code and code needed
        for execution of the program.
    \end{enumerate}

    \section{The Genereral From of a Simple Program}
    Simple C programs have this form: \smallskip

    \textit{directives} 

    \texttt{int main(void)}

    \texttt{\{}

    \tab\textit{statements}

    \texttt{\}}

    \textbf{Directives} - Begin with '\#' symbol, they state what headers
    include to program.

    \textbf{Functions} - They are segments of code that take arguments, and
    returns (or not) a value. Only \texttt{main} function is required.

       
\end{document}
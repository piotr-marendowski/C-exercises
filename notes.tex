\documentclass[12pt, openany]{book}

\usepackage[T1]{fontenc}
\usepackage[utf8]{inputenc}

\newcommand\tab[1][20px]{\hspace*{#1}}      % tabs 20px
\setlength{\parindent}{0pt}                 % tabs at start of the paragraphs

\title{My notes from K.N. King's "C Programming A Modern Approach" 2nd version}
\author{Piotr Marendowski}
\date{March 2023}

\begin{document}
    % PAGE 1
    \maketitle

    % PAGE 2
    \chapter{Note}
    In this material I will go over everything from book,
    trying to summarize every note-worthy subject. I will do
    it, while learning Latex, so good luck to me.

    % PAGE 3
    \tableofcontents

    % PAGE 4
    \chapter{C Fundamentals}
    
    \section{Steps of Executing a C Program}
    Automated proccess:
    \begin{enumerate}
        \item\textbf{Preprocessing} - Preprocessor is executing directives
        (they begin with \#).
        \item\textbf{Compiling} - Compiler translates program into machine
        instructions (object code).
        \item\textbf{Linking} - Linker combines object code and code needed
        for execution of the program.
    \end{enumerate}

    \section{The Genereral Form of a Simple Program}
    Simple C programs have this form: \smallskip

    \smallskip
    \textit{directives} 
    \smallskip

    \texttt{int main(void)}

    \texttt{\{}

    \tab\textit{statements}

    \texttt{\}}
    \bigskip

    \textbf{Directives} - Begin with '\#' symbol, they state what headers
    include to program.

    % PAGE 5
    \textbf{Functions} - They are segments of code that take arguments, and
    returns (or not) a value. Only \texttt{main} function is required.

    \textbf{Statements} - Commands to execute, mostly end with semicolon.

    \bigskip
    \textbf{String literal} - Series of characters enclosed in double quotation marks,
    e.g. \texttt{"Hello world!"}.

    \textbf{New-line character} - \texttt{\textbackslash n} is an escape sequence, which
    advances to the next line of output.

    \textbf{Comments} - Are ommited in program execution, can be used to comment single line
    e.g. \texttt{/* Comment */}, or block of lines. From C99 we can use one line comments
    e.g. \texttt{// Comment}.

    \section{Variables and Assigments}

    \textbf{Variable} - Place to store calculation's output, for using in future.
    \smallskip

    \textbf{Types} - For now, there are two types of variables:
    \begin{itemize}
        \item \texttt{int} - Integer types, can store quite big whole number, but
        that depends on your computer's architecture.
        \item \texttt{float} - Can store bigger numbers, as well as digits after
        decimal point.
    \end{itemize}

       
\end{document}